\documentclass[12pt,a4paper]{article}

\usepackage{polyglossia}
\setdefaultlanguage{english}

\usepackage{fontspec}
\setmainfont{Gentium Basic}

%\linespread{1.3}
\setlength{\parindent}{0em}
\setlength{\parskip}{5pt}

\usepackage{amssymb}
\usepackage{amsfonts}
\usepackage{amsmath}
\usepackage{amsthm}
\usepackage{mathabx}
\usepackage{eulervm}

% theorem, definition, remark
\theoremstyle{plain}
\newtheorem{mytheo} {Theorem}

\begin{document}

\title{Cycle detection problem}
\author{}
\date{}
\maketitle

\section{The problem}

Let function $ f $ maps finite set $ S $ to itself: 
\begin{equation}
f: S \to S
\end{equation}

Choose initial value $ a \in S $ and build a sequence by applying $ f $ iteratively:
\begin{gather}
x_0 = a \\
x_{i+1} = f(x_i), \quad i = 0, 1, \ldots
\end{gather}

Since $ S $ is finite the sequence gets back to the older value:
\begin{equation}
x_{\nu} = x_{\mu} \quad (\nu > \mu)
\end{equation}
and then cyclically repeats values $ x_{\mu}, \ldots, x_{\nu-1} $.

Mathematical details are presented below. The problem in question is to find loop parameters: cycle start index $ \mu $ and its length $ \lambda $.

\section{Mathematical analysis}

Let $ \nu $ be \emph{the largest} index such that values

\begin{equation}
x_0, x_1, \ldots, x_{\nu-1}
\end{equation}
are all different. This means the value $ x_{\nu} $ already appeared in the sequence before, at some index $ \mu < \nu $:

\begin{equation}
x_{\nu} = x_{\mu}
\end{equation}

Let
\begin{equation}
\lambda = \nu - \mu
\end{equation}

By applying function $ f $ to both sides of equation
\begin{gather}
x_{\mu + \lambda} = x_{\mu}
\end{gather}

we have
\begin{gather}
x_{\mu + \lambda + 1} = x_{\mu + 1}
\end{gather}

Apply function $ f $ again:
\begin{gather}
x_{\mu + \lambda + 2} = x_{\mu + 2}
\end{gather}

By induction we conclude
\begin{equation}\label{eq:period}
x_{i + \lambda} = x_{i}
\end{equation}
for any index $ i \geq \mu $.

For future reference let us prove the following statement:

\begin{mytheo}
Given two indices $ i < j $ we have: $ x_i = x_j $ if and only if

\begin{equation}
i \geq \mu
\end{equation}
and
\begin{equation}
\lambda \, | \, j - i
\end{equation}

\end{mytheo}


\section{Floyd's hare and tortoise algorithm}

\section{Brent's algorithm}

\end{document}
